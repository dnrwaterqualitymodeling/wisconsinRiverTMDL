\section{Conclusion}
The methods described here illustrate a preliminary framework used to configure the WRB SWAT model. The parameters calculated for this initial SWAT configuration represent a basic level of detail for known, or well-estimated values that represent physical processes within the basin. These parameters do not fully represent hydrologic processes within the WRB---a number of parameters have been reserved for calibration, or future estimation as we learn more about the hydrologic system during the calibration phase of the project. During the calibration phase, we may find that the model configuration described in this document are inappropriate or inaccurate. Although it is good practice to limit the number of adjusted parameters for the sole purpose of model-fitting, the parameters that were estimated using the methods described in this document may change as we compare the model results to measured data. If these parameters change during the calibration phase, they will likely be adjusted by a scalar to preserve regional patterns and processes.

We made every effort to configure the model in the most accurate and practical way possible. We accounted for the scale of the WRB analysis in the spatial and temporal data configuration, and thus it was necessary to ignore field-scale processes or specific agricultural land management practices and timings. The project team is only mideway throught the complete development and now invite review and comments that may improve  the development and accuracy of the model. 

% As the project team is only midway through the complete development of the WRB SWAT model, we invite comments that may improve our conceptualization of the model. If comments are provided by reviewers, it is important to note that adjustments can only be made if it is clear that the adjustments will not substantially bias the model results. Also to note, recommended adjustments should match the scale of analysis, in geographic and data resolution. For example, we cannot incorporate adjustments that represent processes at the scale of an agricultural field, nor can we incorporate adjustments to highly specific agricultural land management such as specific planting dates during a particularly warm spring.


\pagebreak