\subsection{Streamflow}

After exploring model sensitivity to typically adjusted parameters, we began
calibrating streamflow. Streamflow estimates from SWAT were calibrated by
comparing model estimates to observed flow at 21 USGS gage stations and 3 reservoir spillways operated by WVIC (Table \ref{tab:calibration_sites}). The
objective functions used to assess model fit were Nash-Sutcliffe Efficiency
(\(E_ns\)), percent bias (PBIAS), and (\(r^2\)). Objective functions serve
different purposes in calibrating hydrologic models and as such, we optimized
different objective functions depending on the goal. For example, we aimed to
maximize (\(E_ns\)) when adjusting parameters associated with high flow or
runoff events because an optimized (\(E_ns\)) tends to favor the calibration of
larger values, and we aimed to minimize PBIAS when adjusting
parameters associated with low flow or baseflow because it better represents an overall
water budget and ensures there are no systemic over or under-estimations
(*citation?*).



For all ecoregions, USGS gage sites were used for calculating the objective
function value of model estimates with the exception of the Northern Highland
region. The Northern Highland region contains only two gages that are not
significantly impacted by anthropogenic water regulation. The two remaining gage
sites drain relatively small watersheds, both of which contain a significant
fraction of land that does not contribute to surface runoff. To supplement these
data, we included observed data from 3 reservoir outfalls that were explictly
defined in the model: Willow Reservoir, Rainbow Reservoir, and Rice Flowage. The
reservoir observations that were explicitly incorporated into the model at these
sites should theoretically match perfectly given a well calibrated water budget
and therefore the results of the calibration at these sites cannot be
interpreted as benchmarks for accuracy. However, they provided a means for
assuring that the calibration at the other 2 USGS gage sites were representative
of the general water budget of the whole Northern Highlands region by minimizing
the number of events when the reservoir is estimated to spill over or be
depleted of storage.

\subsubsection{Iterative calibration notes}
\begin{itemize}
    \item Calibrated monthly flows by allowing basin snow parameters to vary
    ]across their full range. Iterated 2000 times. Used PBIAS to select best
    fit.
    Resulting snowfall temperature parameter initially was fit to an unrealistic
    value of -10 degrees C.
    \item Calibrating peak flows
    \begin{itemize}
        \item We calibrated peak flows by adjusting only parameters that control
        runoff and fitting to observed high streamflow data (spring months,
        10\% exceedence).
        During this phase of calibrations, we adjusted the following parameters:
        \begin{itemize}
            \item SURLAG
            \item TIMP
            \item SMTMP
            \item SFTMP
            \item WET\_MXVOL
            \item PND\_EVOL
            \item OV\_N
            \item CH\_N2
% Parameter Name    	t-Stat       	P-Value    
% 8:R__CH_N2.rte    	0.138434890  	0.890009519
% 3:V__TIMP.bsn     	0.293074108  	0.769711578
% 7:V__OV_N.hru     	0.663769054  	0.507457214
% 6:R__WET_MXVOL.pnd	1.186541967  	0.236548353
% 4:V__SURLAG.hru   	-6.730148096 	0.000000000
% 1:V__SFTMP.bsn    	-9.615948651 	0.000000000
% 2:V__SMTMP.bsn    	-11.695267252	0.000000000
% 5:R__PND_EVOL.pnd 	16.020519753 	0.000000000

        \end{itemize}
    \end{itemize}
    \begin{itemize}
        \item We calibrated baseflow by adjusting only parameters that control
        baseflow and fitting to observed low streamflow data (annual 90\%
        exceedence).
        During this phase of calibrations, we adjusted the following parameters:
        \begin{itemize}
            \item WET\_MXVOL
            \item PND\_EVOL
            \item REVAPMN
            \item ALPHA\_BF
            \item GWQMN
            \item GW\_DELAY
            \item GW\_REVAP
            \item RCHRG\_DP
% Parameter Name    	t-Stat       	P-Value    
% 6:R__WET_MXVOL.pnd	0.259973724  	0.795100617
% 5:R__PND_EVOL.pnd 	-1.135580140 	0.257233001
% 8:V__REVAPMN.gw   	1.165684213  	0.244866550
% 1:R__ALPHA_BF.gw  	-1.782857472 	0.075837571
% 4:V__GWQMN.gw     	2.528489334  	0.012079373
% 2:V__GW_DELAY.gw  	4.341652727  	0.000020646
% 7:V__GW_REVAP.gw  	5.657497705  	0.000000042
% 3:V__RCHRG_DP.gw  	-10.791636578	0.000000000
        \end{itemize}
    \end{itemize}
    \item We then reassigned parameter values for those parameters in the both
    the spring runoff and baseflow calibrations that met a significance level of
    p < 0.01. With this reassignment, we calibrated curve number for summer
    months for flows above the 25\% exceedence flow.
\end{itemize}
\pagebreak
