\subsection{General Methods}

The primary software used for calibrating flows (discharge) and in-stream
pollutants was SWAT-CUP 2012 v.5.1.6.2 (SWAT Calibration and Uncertainty
Programs). SWAT-CUP allows users to adjust parameters in various ways:

\begin{itemize}
  \item Adjustment type
  \begin{itemize}
      \item Relative adjustment by a scalar
      \item Absolute adjustment by addition or subtraction
      \item Replacing default values
  \end{itemize}
  \item Adjustment filters
  \begin{itemize}
      \item Hydrologic Soil Group
      \item Soil Texture Class
      \item Land-use
      \item Subbasin
      \item Slope
      \item Other manual or text-file-specific conditions
   \end{itemize}
\end{itemize}
	
SWAT-CUP software was chosen because it is relatively easy to set up and
understand, it is flexible enough for most users with typical SWAT projects, and
it offers a parallel processing module for users that are calibrating large
projects with many HRUs. We chose to use the parallel processing module because
of a substantial reliance on auto-calibration.
Auto-calibration was required due to the large number of HRUs in the WRB and the
number of calibration sites for streamflow, sediment, and phosphorus that span a
wide range of land-use, soil, and topographic geographies. Because we chose to
run SWAT scenarios in parallel using SWAT-CUP, we were committed to using the
SWAT-CUP-specific SUFI-2 algorithm (Sequential Uncertainty Fitting), which is
non-iterative or convergent within a set of simulations (i.e. the parameter
adjustments of each subsequent model run do not rely on objective function
values of any prior model run) and is thus amendable to parallel
computing (*citation for SUFI2*).

Because we relied heavily on auto-calibration using parallel processing, we
required computing resources with many processors, enough random access
memory (RAM) to support numerous runs across processors, and disk storage with
short access times and minimized latency. We chose to rent cloud computing from Amazon Web Services
Elastic Cloud Computing (EC2) service. All calibration computing for the WRB
SWAT model was executed on an EC2 instance with 16 dual-core Intel Xeon
E5-2680 processors, 60 gigabytes of RAM, and a solid-state drive with 750
general purpose Input/Output Operations per Second (IOPS) and 3000 burst IOPS.

Even after reducing the number of HRUs by aggregating land-use, soil, and slope
classes, model run time was insufficiently slow for auto-calibration. For the
purpose of debugging, exploratory analysis, and general calibration, the full
model with 5,351 HRUs was reduced to 1,651 by increasing percent thresholds for
preserving land-use, soil, and slope classes to at least 5, 50, and 50\%
respectively (discussed in more detail in Section \ref{sec:hru_definition}). The
1,651-HRU model was only used for narrowing the ranges of parameter adjustments,
whereas the full model was used to calibrate the final parameter values and
the model uncertainty associated with those parameter ranges.

Because hydrologic properties vary widely across the WRB, we chose to split
parameter sets by geographical filters that were known \textit{a priori} to
bound similar hydrologic properties. First, we split USGS gage sites, and the
SWAT subbasins associated with them, by an edited version of Ecological
Landscapes of Wisconsin (*Cleland et al 1997*
http://dnr.wi.gov/topic/landscapes/documents/StateMaps/Map_S1_ELs.pdf).
Ecological Landscapes were lumped together in some cases to improve the
parsimony of the overall model and simplify calibration. Lumping was guided by
visualizing similarity of hydgraphs of nearby USGS gage sites. Specifically, the North
Central Forest and Forest Transition landscape boundaries (referred to
hereafter as the Northern Forest), and the Central Sand Plains and Central Sand
Hills (referred to hereafter as the Central Sands) regions were aggregated
for a total of 4 discrete zones within which all parameter adjustments would be
equal with the exception of two additional splits on A/B and C/D hydgrologic
soil groups (HSG) in the Northern Forest and Central Sands regions. In these two
regions, the Western half of the region is dominated by C/D type soils and the
Eastern half of the region is dominated by A/B type soils, which provided an
effective method for explaining differences in the hydrographs of USGS gage
sites on either side of the HSG divide.

Calibration of a SWAT model is rarely executed using all model data
simultaneously with all calibration data sets nor is done linearly from start
to finish. Rather, it is done iteratively and piecemeal as new information
is gained about the model's response to individual or multiple simultaneous
parameter adjustment. However, general workflow was followed for
calibrating the WRB SWAT model that follows guidance in SWAT literature. First,
daily streamflow was calibrated within each Ecological Landscape where the
effects of parameters adjustments to subbasins within neighboring Ecological
Landscapes would be minimized to isolate the effects of parameter adjustments
within the Ecological Landscape being calibrated.

\begin{itemize}
    \item Streamflow
    \begin{itemize}
        \item First,
    \end{itemize}
    \item Sediment
    \item
\end{itemize}

\pagebreak