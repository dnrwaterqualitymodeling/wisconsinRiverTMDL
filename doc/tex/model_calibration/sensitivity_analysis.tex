\subsection{Sensitivity Analysis}
	A literature review was conducted to assess which and how SWAT parameters were adjusted in other projects. Before pursuing calibration we carried out a one-at-a-time sensitivity analysis for the parameters found in the literature as well as others thought to be of interest in calibrating the model. By one-at-a-time sensitivity analysis we mean that the model was run a set number of times, in this case, for twenty-five iterations. For each of these iterations only one parameter was adjusted, and the model output collected. In this way the effect of one parameter could be isolated and its impact on the model assessed. 

	For each parameter minimum and maximum parameter values were set, from which parameter ranges were created; the parameter ranges defined the increments upon which the iterations were run. The minimum and maximum parameter values were determined based upon the literature review and what parameter values were deemed realistic. For example, parameters affecting snowmelt can be set to melt snow at -20\degree C  to 20\degree C, but it was thought realistic to set the minimum and maximum parameter values to -2\degree C  and 2\degree C, rather than allowing snow to melt at -20\degree C  or freeze at 20\degree C. Some parameters were adjusted absolutely and others were adjusted relatively, depending on how much was known about their response or if the parameters varied geographically. For example, snow melt characteristics were not thought to vary throughout the basin and so they were set to the same value. The parameter CN\_OP, which adjusts the curve number based upon a land management operation event, was adjusted relatively because different management operations had different curve numbers; thus their ranges were defined based a percentage (e.g., plus and minus five percent of the initial value). The parameters included in the senstivity analysis are shown in Table \ref{tab:sens_anl}.

	Sensitivity was assessed by calculating the change in model output for each subbasin from one iteration to the next. This was accomplished by subtracting the subsequent iteration from each iteration and then taking the mean of the absolute values of these differences.

	This sensitivity analysis was coded in R and setup to run on multiple processors at once.

\pagebreak