\subsection{Sensitivity Analysis}
A literature review was conducted to assess which and how SWAT parameters have
been adjusted in other projects. We carried out a one-at-a-time sensitivity
analysis for the parameters found in the literature review as well as others
thought to be of interest in calibrating the model. By one-at-a-time sensitivity analysis
we mean that the model was run a set number of times, in this case, for
twenty-five iterations and for each of these iterations only one parameter was
adjusted, and the model output collected. In this way the effect of one
parameter could be isolated and its impact on the model assessed.

For each parameter minimum and maximum values were set, from which parameter
ranges were created; the parameter ranges defined the increments upon which the
iterations were run. The minimum and maximum parameter values were determined
based upon the literature review and what parameter values were deemed
realistic. For example, parameters affecting snowmelt can be set to melt snow at
-20\degree C  to 20\degree C, but it was thought more realistic to set the
minimum and maximum parameter values to 0.5\degree C  and 2.5\degree C. Some
parameters were adjusted absolutely and others were adjusted relatively,
depending on how much was known about their response or if the parameters varied
geographically. The parameter ALPHA\_BF, which adjusts the amount of baseflow to
a reach was regionalized based on regression analysis and so varied throughout
the basin; thus ALPHA\_BF ranges were defined based on a percentage (i.e., plus
and minus five percent of the initial value). The parameters included in the
sensitivity analysis are shown in Table \ref{tab:sens_anl}.

During the sensitivity analysis, after each model run the model output was
collected. The output that was collected varied based on which parameter was
being run. For groundwater, soil, management, HRU, and basin parameters water
yield was collected, found in the output.sub output file. This output reports
the amount of water generated by the land area in the subwatershed. Reach
parameters do not have an affect upon the amount of water leaving the landscape
and so streamflow information from the output.rch file was used to assess how
changes in these parameters affect the model.
	
Sensitivity was assessed by calculating the change in model output for each
subbasin from one iteration to the next. This was accomplished by subtracting
the subsequent iteration from each iteration and then taking the mean of the
absolute values of these differences. This value then represents the mean annual
change in model output expected from a change in target parameter value.

	% This sensitivity analysis was coded in R and setup to run on multiple
	% processors at once.

\pagebreak