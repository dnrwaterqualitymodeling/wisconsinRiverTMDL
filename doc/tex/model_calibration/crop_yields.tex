\subsection{Crop Yields}

Crop yields were calibrated before flow and pollutants because crop yields give an indication of addequate biomass and ground cover produced, which in turn affects the amount of runoff. Crop yield calibrations were not carried out with SWAT-CUP because there are no routines to aggregate HRU-level yield output. Following \citet{baumgart_source_2005} crop yields were calibrated entirely using the BIO\_E, the radiation-use efficiency or bioimass-energy ratio, parameter. This parameter controls plants efficiency of converting radiation into biomass, thus by adjusting this parameter, we matched modeled crop yields with observed crop yields.

Observed data consisted of crop yield data acquired from the National Agriculture Statistics Serivice (NASS), from the county survey data, (\footnote{\url{http://quickstats.nass.usda.gov/results/CD8890FD-566F-3F66-8C8A-CB932E358991}} retrieved on 30 September, 2014. The values used were corn grain--yield, bu/acre, soybean--yield, bu/acre, and Hay/Haylage, Alfalfa - Yield tons/acre, dry basis. ``Hay/Haylage, Alfalfa - Yield tons/acre, dry basis'' was chosen over ``Hay, Alfalfa'' because it was thought that ``Hay \& Haylage, Alfalfa'' encompasses the former, and so would be a better estimate of total alfalfa yield as modeled by SWAT. The observed data was based on county averages so we aggregated this information into an area weighted average so we had a basin-wide value from which to compare. Comparisons between observed and simulated were also assessed on a county-level average.

% From James Johanson(?) the Alfalfa Specialist at NASS, it was learned that NASS alfalfa yields are annual totals, not average cutting yield. For comparison with SWAT, it is necessary to make sure that annual alfalfa yield is found. We found that merely summing all months yield (with monthly output) did not give accurate values of average annual yield. Use the row corresponding to HRU average annual.

% Note: Do not use HRU average yield over the entire modeling period, as SWAT
% averages the yield over the modeling period ACROSS crops. Instead, select
% average annual HRU yield.

County surveys conducted by NASS consist of surveying farm operators about aspects of the operation and farm. This means that yield moisture weights are not necessarily standard. NASS assumes that corn, soybean and hay/haylage are 15\% 12.5\% and 13\% moisture, respectively; these values were corrected to match SWAT's output, which is in dry matter, 0\% moisture. 
% Currently (1 October) we are assuming that SWAT reports yield and biomass in 0\% moisture dry weight, as do \citet{almendinger_contructingsunrise_2010} though other sources \citep{srinivasan_swatungauged_2010} say that SWAT reports in 20\% moisture.


The calibration efforts were directed at alfalfa, corn grain, and soybean yield and centered on adjusting the BIO\_E to match observed data to simulated data. 
% Corn silage was initially considered for calibration, but there was too much uncertainty in the way that the NASS silage weight was reported. 
SWAT was run with a BIO\_E of 10 through 90, the largest possible range of values, by increments of 15. After every run, the modeled yield was obtained from the output.hru file. A basin-wide average yield was calculated using all the HRUs in which that crop was grown, weighted based on the area of the HRU. The simulated basin-wide average plotted against the BIO\_E value to create a calibration curve, which illustrated how yield changed with adjustments to BIO\_E; in all cases yield increased linearly with an increase in BIO\_E. A simple linear regression model was fit with BIO\_E as the response variable and yield as the explanatory variable. With this model, we used the observed basin-wide yield and from this predicted the optimum BIO\_E.  

Calibrated BIO E values are given in the following table. 

\begin{table}[h!] \centering
	\begin{tabular}{l c c}
		\hline
		Crop & Default BIO E & Fitted BIO E \\[0.25ex]
		\hline \hline
		Corn & 39 & 31.86 \\
		Soybean & 25 & 49.00 \\
		Alfalfa & 20 & 10.00 \\
		\hline

	\end{tabular}
\end{table}	
	%As of 1 October 2014, crop yield calibration focused on adjusted the BIO\_ E parameter, following other workers in the region \citep{baumgart_lowerfox_2005, almendinger_contructingsunrise_2010}. This is found in the plant.dat input file. 
\pagebreak