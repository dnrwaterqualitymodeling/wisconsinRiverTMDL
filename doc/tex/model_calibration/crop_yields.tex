\subsection{Crop Yields}
Crop yield calibrations were not carried out with SWAT-CUP because there are no
routines to aggregate HRU-level yield output.
	\subsubsection{County Survey}
Observed data consisted of crop yield data acquired from the National
Agriculture Statistics Serivice, from the county survey data,
(\href{http://quickstats.nass.usda.gov/results/CD8890FD-566F-3F66-8C8A-CB932E358991}{from
the quick stats utility}), retrieved on 30 September, 2014. For more information
on county surveys
\href{http://www.nass.usda.gov/Surveys/Guide_to_NASS_Surveys/County_Agricultural_Production/index.asp}{click
here}. The stats used were corn grain- yield, bu/acre, soybean - yield, bu/acre,
and Hay/Haylage, Alfalfa - Yield tons/acre, dry basis. The first two are
self-evident, but the third was chosen over ``Hay, Alfalfa'' because it was
thought that ``Hay \& Haylage, Alfalfa'' encompasses the former, and so would be
a better estimate of total alfalfa yield.
	
From James Johanson(?) the Alfalfa Specialist at NASS, it was learned that NASS
alfalfa yields  are annual totals, not average cutting yield. For comparison
with SWAT, it is necessary to make sure that annual alfalfa yield is found. We
found that merely summing all months yield (with monthly output) did not give
accurate values of average annual yield. Use the row corresponding to HRU
average annual.
	
Note: Do not use HRU average yield over the entire modeling period, as SWAT
averages the yield over the modeling period ACROSS crops. Instead, select
average annual HRU yield.

	\subsubsection{Unit Conversions}
County surveys conducted by NASS consist of surveying farm operators about
aspects of the operation and farm. This means that yield moisture weights are
not necessarily standard. NASS assumes that corn, soybean and hay/haylage are
15\% 12.5\% and 13\% moisture, respectively.
Currently (1 October) we are assuming that SWAT reports yield and biomass in 0\%
moisture dry weight, as do \citet{almendinger_contructingsunrise_2010} though
other sources \citep{srinivasan_swatungauged_2010} say that SWAT reports in 20\%
moisture.

	\subsubsection{Calibration}
The calibration efforts were directed at alfalfa, corn grain, and soybeans. Corn
silage was initially considered for calibration, but there was too much
uncertainty in the way that the NASS silage weight was reported. As of 1 October
2014, crop yield calibration focused on adjusted the BIO\_ E parameter,
following other workers in the region \citep{baumgart_lowerfox_2005,
almendinger_contructingsunrise_2010}. This is found in the plant.dat input file.
This parameter can be a range from 10 to 90. SWAT was run with a BIO\_E of 10
through 90, by increments of 5. After every run, the output.hru file was
analyzed and basin-wide average yields were calculated. From this a calibration
curve was derived, and a model fit. The mean, basin-wide observed average was
then used to obtain a fitted BIO\_E. Associated images can be found in the
validation folder ``T:/Projects/Wisconsin \_ River/Model \_ Documents/TMDL \_
report/figures''. Calibrated BIO E values are given in the following table.
\begin{table}[h!] \centering
		\begin{tabular}{l c c}
			Crop & Default BIO E & Fitted BIO E \\[0.25ex]
			\hline \hline
			Corn & 39 & 31.32 \\
			Soybean & 25 & 54.65 \\
			Alfalfa & 20 & 9.72 \\
			\hline

		\end{tabular}
	\end{table}	
\pagebreak