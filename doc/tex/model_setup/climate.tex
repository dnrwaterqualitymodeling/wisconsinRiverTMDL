\subsection{Climate and Weather Data}\label{sec:climate_data}
%Provided are data concerning temperature (found in the "tmp" folder) and precipitation	(found in the "pcp" folder). The data are in two different formats, text files (.txt) and data base files (.dbf),	note that these files are, to some extent, redundant. The data in the database files are the raw data from the climate stations formatted for input to the SWAT model. The naming convention for the DBF file names is the weather station ID followed by a 'tmp' for temperature or 'pcp' for precipitation.	Similarly, the naming convention for the text files is a 't' for temperature or 'p' for precipitation, followed by the station ID. The files pcp.txt and tmp.txt contain the station names, and the latitude and longitude in WGS84 coordinate reference system.

Daily weather and climate data used in the SWAT model were downloaded from the National Climate Data Center (NCDC) - Global Historic Climate Data Network website. Weather data include precipitation and temperature.  Data sets from 120 different stations were downloaded, from which precipitation data was taken from 74 stations, and temperature from 46. Climate data included solar radiation, wind speed, and relative humidity.   

Not all stations had complete data records covering the entire timeframe being modeled.  For time periods missing precipitation or temperature data, the record was supplemented with data from the nearest weather station that did have data for that time period, see Figure \ref{fig:climate_missing} for locations of precipitation and temperature stations and the number of missing days for each station.   For time periods missing data at all stations, weather data was generated automatically by SWAT's internal weather generator.  SWAT's weather generator was was used for all missing solar radiation, wind speed, and relative humidity records. 
