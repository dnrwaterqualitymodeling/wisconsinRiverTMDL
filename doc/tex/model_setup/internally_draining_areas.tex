\subsection{Wetlands}\label{sec:wetlands}

	SWAT considers wetlands in a manner very similar to how it considers ponds, the difference only being one additional parameter in ponds, NDTARG. There were several parameters that needed to be calculated for the basin's wetlands and these were the same as for ponds: the fraction of each subbasin that drains to wetlands, the normal and maximum surface areas, and the normal and maximum volumes. 
	
	These parameters were calculated for each subbasin using a terrain-based approach. A digital elevation model (DEM) was filled using the Fill function in ArcGIS, filling all of the sink areas and causing all simulated water to run off of the landscape. The original DEM was subtracted from the filled DEM to derive a surface of the depth of internally drained areas or sinks.  This sinks layer shows the internally drained areas for the basin.
	
	The areas classified by the CDL as herbaceous wetlands, woody wetlands, and cranberries were considered to be areas where wetland vegetation is likely to be found. If wetland vegetation exists it can be assumed that the landscape has a consistent wetland hydrology, enough that it is expressed in the vegetation. The intersection or overlap of the sinks layer and the wetland vegetation, as identified by the CDL, was considered to be the principal wetland surface area. To calculate principal storage volume, we assumed an average water depth for any given wetland to be 0.5 m, and then multiplied this by the principal surface area. For emergency surface area, we used the union of CDL wetlands and sink areas. The emergency storage volume was calculated by summing the volumes of sinks and adding that to the principal storage volume. The maximum wetland surface area was divided by the subbasin area to derive the fraction of the subbasin that contributes to wetlands. Contributing areas to wetlands can be seen in Figure \ref{fig:wetlands_and_ponds}.

	There are precedents to using a terrain-based approach to defining wetland areas in SWAT. \citet{almendinger_constructing_2007} considered internally drained areas as wetlands (as identified by remote sensing\footnote{Specifically, the remotely sensed imagery was from the WISCLAND data set; a dataset of landcover determined from LANDSAT imagery.}) if they were not connected to the main channel and lakes were considered ponds in their SWAT model. Wetlands, identified through remote sensing, were considered SWAT wetlands only if they occur on the main channel. Similarly, \citet{kirsch_predicting_2002} considered internally drained areas as wetlands in SWAT if they overlapped with remotely-sensed-defined wetlands; if they did not, they were considered ponds.  \citet{almendinger_constructing_2010} modeled closed internal depressions as wetlands and open (those draining to the main channel) as ponds.



