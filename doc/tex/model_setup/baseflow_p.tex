\subsubsection{Baseflow Phosphorus} \label{sec:gwp}
	The baseflow component of background phosphorus is not simulated by default in SWAT and so the parameter controlling soluble phosphorus in groundwater (GW\_SOLP) needs to be populated manually. Rather than estimating a value of baseflow phosphorus concentrations for the whole WRB, we attempted to regionalize this parameter. In the Wisconsin River Basin SWAT model, we used values of reference baseflow phosphorus from a USGS study of nutrient concentrations in wadeable streams in Wisconsin \citep{robertson_wadeable_2006}. In their study, \citet{robertson_wadeable_2006} used a multiple linear regression equation to predict reference phosphorus in nutrient boundaries known as ``environmental phosphorus zones'', which they use as a way of dividing the state into smaller homogeneous regions. These zones were derived in an earlier study by \citet{robertson_phosphoruszones_2006}.

	
	For each of the phosphorus zones, the percent land area in agricultural use and percent land in urban use was calculated along with the number of point sources. Using these variables, a multiple linear regression model predicting the concentration of phosphorus was fitted. To represent the scenario where human impact is negligible, the values of the predictors of this model (all of which represent human impact) are set to zero. With the predictors set to zero (i.e., model intercept), the predicted value of the model is the median phosphorus concentration when human impact is zero (eq. \ref{eq:gw_p}). The SWAT subbasins were overlaid with the phosphorus zones, and the predicted reference phosphorus was calculated for each subbasin using an area-weighted average. These reference phosphorus levels were input into SWAT using the groundwater soluble phosphorus parameter.
	
	
	\begin{equation}
	\bm{P} = e^{\beta_0 + \beta_1 \bm{A} + \beta_2 \bm{U} + \beta_3 log_{10}(\bm{O})}
	\label{eq:gw_p}
	\end{equation}
		
	Here, $\bm{A}$, $\bm{U}$, and $\bm{O}$ are the percent of the watershed in agricultural and urban landuse and the number of point source outfalls, respectively. When these are set to zero, the equation reduces to simply:
	
	\begin{equation}
	\bm{P} = e^{\beta_0}
	\end{equation}
	
	The resulting distribution of the groundwater phosphorus in the WRB can be found in Figure \ref{fig:groundwater_p}.
	
	\citet{robertson_wadeable_2006} intend their background phosphorus values to estimate median phosphorus concentration in streams when there is no human impact in the watershed. They do not specifically estimate the groundwater or baseflow contribution to reference phosphorus concentration. We assume that the median reference phosphorus estimate is an accurate estimate of baseflow phosphorus concentration because a landscape under natural conditions (that is one without human impact) will experience much less runoff, and that the median estimate represents low-runoff conditions \citepalias{nrcs_tr55_1986}. Due to a lack of appropriate monitoring data, we are left to assume that this is our best estimate of baseflow phosphorus. However, as the development of the model progresses, we may discover patterns in the output phosphorus loads that provide insight into the quality of this assumption.
