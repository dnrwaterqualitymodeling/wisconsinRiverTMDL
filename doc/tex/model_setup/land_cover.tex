\subsubsection{Land Cover}\label{sec:land_cover}
The composite land cover developed for the SWAT model input began with the United States Department of Agriculture (USDA) National Agricultural Statistics Service (NASS) 2011 Cropland Data Layer (CDL) for Wisconsin \cite{usda_cdl_2011}. The layer, originally created to provide agricultural information for the major crops to the USDA Agricultural Statistics Boards, provides a raster-based, geo-referenced data layer that defines growing-season land cover at a cell resolution of 30x30m for Wisconsin using satellite imagery from a variety of satellites (USDA, 2011). For non-agricultural land cover, the USDA NASS CDL relies on the United States Geological Survey (USGS) National Land Cover Database (NLCD) 2006. The 2011 USDA NASS CDL was selected because that year had improved accuracy statistics when compared to other years, and there were no flooding or drought events within the growing season. To improve the landcover definition, Wisconsin Wetlands Inventory (WWI) information was integrated into the 2011 CDL. The WWI coverage provides the geographic extent of wetlands that have been digitized from aerial photography, verified through photo interpretation, and compared against soil surveys, topographic maps, and previous wetland inventories (WDNR 1991).