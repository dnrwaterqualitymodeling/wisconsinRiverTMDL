\subsection{Subbasin Delineation}\label{sec:sub_delineation}
The first step in configuring a SWAT is delineating subbasins. The Wisconsin River is a relatively large area for a TMDL project. Consequently, much of the point and non-point load-reduction efforts will occur as nested projects within the overall TMDL framework.
Hydrologic and regulatory transitions were used to guide the placement of TMDL subbasin transitions. TMDL subbasins were delineated: 
\begin{enumerate}
	\item to address specific water quality impairments where local water quality does not meet codified standards. Consideration was given to streams that are likely to be impaired, but where sufficient monitoring data do not currently exist.
	\item near point source outfalls. Delineations were not required to be at precisely the location of the outfall; if we could assume that streamflow does not significantly increase between the discharge location and the next downstream subbasin division, it was not necessary to further subdivide the subbasin at the discharge location.
	\item at locations where water quantity and quality were measured during the model period for use in model calibration.
	\item at major transitions of water quality standards, for instance at river impoundments that receive lake criteria.
	\item at major hydrologic transitions such as the confluence of two large streams or where there are significant changes in landuse/landcover.
\end{enumerate}

Beyond the above criteria, we made an effort to subdivide the remaining subbasins so that the resulting subbasins are of relatively homogenous area---having similarly sized subbasins results in more accurate timing of peak flows during runoff events.

After the locations of subbasin outfalls were identified, we delineated the contributing area upstream of each.
ArcSWAT will automatically delineate model subbasins, but because of our specific reasons for creating subbasins outlined above, we manually created the SWAT subbasins by aggregating the WHDPlus dataset \citep{wdnr_whdplus_2013}, which contains subcatchments within the Hydrologic Unit Code (HUC) 12 basins (which are standard watersheds created and maintained by the United States Geologic Survey [USGS]). 
We delineated 338 subbasins with an average size of 68 km$^2$ (standard deviation = 80 km$^2$); larger subbasins were located in areas with fewer water quality impairments and points sources. This size is smaller than the average HUC12 watershed (84 km$^2$), which is the scale at which TMDL implementation strategies are typically evaluated.