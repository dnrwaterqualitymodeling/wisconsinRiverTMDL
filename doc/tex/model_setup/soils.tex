\subsection{Soil Data Aggregation}\label{sec:soils}

Soils are a critical part of the SWAT modeling framework; they determine many
surface hydraulic properties such as texture, hydraulic conductivity, and
available water capacity. We used the county-scale Soil Survey Geographical
Database (SSURGO) \citepalias{staff_gridded_2014}. For more information about SSURGO
data see the SSURGO metadata\footnote{\url{http://www.nrcs.usda.gov/wps/portal/nrcs/detail/soils/survey}}.
The SSURGO database is structured in three levels of information: map units,
components, and horizons (Figure \ref{fig:ssurgo_schematic}). Horizons are the fundamental unit of soil in SSURGO, and are therefore where the majority of soil information is stored in the database.
Components are aggregations of horizons that represent a full soil profile,
typically conforming to the Official Series Description (OSD). Map units
are discrete polygons drawn on a map (originally mapped at scales from 1:12,000
to 1:63,360) that contain one or more components that are stored non-spatially
in the database---that is, only a list of components and their percent
composition of the map unit is given. 

We chose to use the gSSURGO distribution of SSURGO. gSSURGO is a form of the SSURGO database that is packaged in a more convenient form for coarse scale projects such as the WRB TMDL. The tabular data representing the components and
horizons were joined together so that each component had the data required for
the SWAT model (Table \ref{table:soil_attr_in_swat}). For all these properties, the representative value given by
SSURGO was used.

HRU definition in a SWAT model is a balance of incorporating the most important
pieces of information without overloading it with redundant or insignificant
information---a modeler should represent every process that controls the system,
however an overloaded model requires more computational resources, which may not
be feasible to acquire. To reduce the number of HRUs in the model and generally create a 
simpler and more efficient model, we aggregated
soils together based on similarity of several key properties that impact the
hydrologic cycle. This was a two step process: first, components within map
units were aggregated
together,\footnote{\url{https://github.com/dnrwaterqualitymodeling/wisconsinRiverTMDL/blob/model_setup_public/soils/step1_aggregate_gSSURGO.R} } 
and second, map units were aggregated together based on
similarity\footnote{\url{https://github.com/dnrwaterqualitymodeling/wisconsinRiverTMDL/blob/model_setup_public/soils/step2_aggregate_gSSURGO.R}} (Figure \ref{fig:agg_flow_diagram}).

Several changes were made to the dataset before aggregation, in order to
facilitate processing. Soil organic carbon content is required by SWAT, but is
given by SSURGO as soil organic matter. The percent organic matter value given in SSURGO
was converted to percent organic carbon by multiplying by 50\% which is the generally accepted average carbon content of soil organic matter \citep{brady_elements_2004}. The HSG is
denoted as a letter in SSURGO, either A through D, or if the soil has different
characteristics when drained, as two letters, A/D, B/D or C/D, the latter of
which is the natural state of the soil if not artificially drained (i.e.,
through tiling or ditching) while the former is if the soil is drained.
In order to average the different components it was necessary to convert these letters
into numbers; groups A through D were converted to 1 through 4 to correspond
with increasingly wetter drainage conditions. Once a number was obtained for the
HSG, it was treated as any other soil property in the aggregation process,
then rounded to the nearest integer, and converted in the same manner to a letter
once the aggregation was finished. For those components with dual HSGs, we
assumed that if half of the area in the map unit was agriculture it was drained
and the first HSG taken. Conversely, if the land use was not majority agriculture
then it was assumed to not be drained and the ``D'' designation was chosen.

The first aggregation step was to aggregate components by map unit to conform to
the SWAT soils data structure. The data structure for soils in SWAT does not
directly conform to SSURGO data structure, mainly that there is no analogue to the
SSURGO \textit{component} level in SWAT---in other words, soils in SWAT cannot
be subdivided (Figure \ref{fig:component_schematic}). 
Similar to \citet{gatzke_aggregation_2011}, who aggregated SSURGO data for a SWAT model, 
we aggregated components by computing
component-weighted averages of each soil property for any given depth of soil from the soil surface to the average depth.
These averages were computed using the \texttt{slab} function in the \texttt{aqp} package in R \citep{beaudette_algorithms_2013}. We
used this algorithm to apply a depth weighted average to each horizon, while
also weighting the percent composition of each component. The depth and number of horizons of
the aggregated soil profile produced by this algorithm must be specified before
processing. The depth was calculated as a weighted mean of the full depths of
soil profile in each of the components, with the weights equal to the percent composition of each
component. As the number of horizons was assumed not to matter as much as the
maximum depth, an arbitrary number of five horizons was chosen for the
aggregation algorithm.

Using the above aggregation method 48,585 individual soil components were
aggregated to 1,603 map units. Because the HRU used in
SWAT is derived using unique combinations of land use, slope and soil types,
this number of soil map units is still too many for efficient computation  and so
the second step of the soils data configuration was necessary to further reduce
the number of soil types.

Other researchers have aggregated soil types by their taxonomic class
\citep{gatzke_aggregation_2011} but Soil Taxonomy, the soil classification system of the US, primarily classifies based on soil morphology and not
necessarily on properties relevant to SWAT. We decided that the most relevant soils
information to SWAT is hydrology data, specifically the HSG, which has a large impact on the soil curve number. With this
consideration, aggregation was based around (and so preserved) the HSG of the
map unit. Groups of the same HSG were divided into smaller groups (hereafter
known as clusters) of homogeneous soil properties. The map units within each of these clusters were then averaged together to create an average profile for that homogeneous set of soils. These averages were then
used as the soil types for the HRU definitions and the SWAT modeling.

To begin, each map unit (each of which is an aggregation of components, as described above)
 was placed into one of four groups according to its hydrologic soil
group, A, B, C or D. To subdivide these groups further, a clustering algorithm
was used to objectively create clusters of map units with
homogeneous soil properties. For this purpose, we used Gaussian mixture models to
assign map units to clusters. The mixture model approach we used was implemented within the
\texttt{Mclust} function in the \texttt{mclust} package in R \citep{fraley_mclust_2012}.
A mixture model is a probabilistic model for representing the presence of
subpopulations within an overall population. In our case, the overall population
would be the group of map units of similar hydrologic soil groups (i.e., all map units
with an HSG of A), while the (unknown) subpopulations are the clusters of
map units with similar distributions of soil properties (such as a clusters of
sandier soils, shallow soils, or slow saturated conductivity). Using the default
settings of the function, we clustered all of the A HSG map units into
6 clusters, B HSG map units had 8 clusters, and C and D both had 9 clusters.

In order to use the clustering function we had to put our data in a format in which it could 
be used by the \texttt{Mclust} function. We calculated horizon depth-weighted averages of each soil property for each map unit, essentially collapsing the soil profile down to one aggregate horizon with average properties. Profile depth was still
considered in the clustering algorithm using the total depth of the profile.

After each map unit had been assigned to a cluster, the map units within each cluster were
aggregated together to form a composite or average soil profile. 
The same soil profile aggregation algorithm \citep{beaudette_algorithms_2013}
used to aggregate several components together in the first step was used to combine the soil profiles of a cluster into one
composite soil profile. In this implementation, each map unit was given equal
weight in the aggregation algorithm. 

Not every map unit was included in the clustering procedure. Several of the soil property fields of the SSURGO dataset were not populated or commonly had ``no data'' values, these properties were not used in the
clustering process so the spurious zeros would not influence the algorithm (Table \ref{table:soil_attr_in_swat}). These properties were coarse fragments, calcium carbonate, and electrical conductivity. Albedo and pH were also excluded from the clustering algorithm. Mapunits that had no HSG designation were not included, nor were map units that did not have information on the soil properties of the horizons. Examining these excluded map units revealed that they were generally disturbed landscapes or those without a significant soil layer such as pits, landfills, urban or made land, rock outcrops, and water. These miscellaneous map units were all grouped together as one cluster with the exception of water. All water map units were collapsed into one using properties described in the default ArcSWAT SSURGO database\footnote{\url{http://swat.tamu.edu/media/63316/SWAT_US_SSURGO_Soils.zip}}. 

A total of 35 soil classes were distilled from this process. An example of the properties of each cluster are shown in Table \ref{table:soil_prop} and the number of map units in each cluster can be found in Table \ref{table:soil_clust}. The hydrologic soil groups and the clusters within these groups are displayed in Figure \ref{fig:soil_boxplots}, which shows the relative variability of soil properties of map units that were aggregated to clusters.

