\subsubsection{Soils}
Original thinking was that the State Soil Geographic (STATSGO) database coverage would be sufficient for the scale of the WRB. After considering that the elevation and landcover data was at a finer scale than STATSGO and that soils data was available at a finer resolution, it was decided to use the county level soils data, Soil Survey Geographic (SSURGO) database, in the SWAT modeling. (Are there other issues with STATSGO that justify this decision? Read Mednick and other papers.)  		

Soil data configuration is conceptualized as consisting of two parts, both involving aggregating and combining soil data. The first part is aggregating the soils data to the mapunit level, while the second involves further reducing the number of soil types by aggregating mapunits together.		

There are often several different soil types, usually soil series, mapped within a mapunit and thus it is necessary to combine or aggregate these in order to have a representative soil profile for each mapunit (as required by SWAT), that is each soil will have data for each of the horizons; this process was carried out in the script ``step1\_aggregate\_gSSURGO.R''. There are more than 1500 SSURGO mapunits within the WRB and if all of these were used in SWAT it would result in many tens of thousands of HRUs and so necessitating a very long computing time per model run, complicating the calibration process in which thousands of model runs are necessary. To reduce the number of mapunits further aggregation was necessary, this process was carried out in the script ``step2\_aggregate\_gSSURGO.R''.
 
The starting point for aggregating the statewide soils data was the gSSURGO dataset (downloaded on 29 October, 2014). This dataset is intended to be primarily a gridded version of the SSURGO data for a state. Also included are the tabular data with soil properties and the vector polygons for all of the soils units from which the gridded data was created. The tabular data representing the components, horizons, and the horizon fragments were joined together so that each component had the data necessary for the SWAT model; these properties were hydrologic soils group, albedo, horizon depths, bulk density, available water capacity, organic matter, saturated conductivity, total percentage of clay, silt and sand, K factor, electrical conductivity, calcium carbonate concentrations, pH (1 to 1 in water), and coarse fragment percentage. Of these, these hydrologic soil group and albedo were given for at the component level, while the others were given for each horizon.  For all these properties the representative value given by SSURGO was used. For more information about these parameters see the \href{http://www.nrcs.usda.gov/wps/portal/nrcs/detail/soils/survey/?cid=nrcs142p2_053631}{SSURGO metadata}. 

Several changes were made to the dataset before aggregation in order to facilitate processing. Soil organic carbon content is required by SWAT, but is given by SSURGO as soil organic matter. The organic matter value given in SSURGO was converted to an organic carbon value by multiplying by the average carbon content of soil organic matter, 50\% \citep{brady_introsoils_2002}. The hydrologic soil group (HSG) is denoted as a letter in SSURGO, either A through D, or if the soil has different characteristics when drained, as two letters, A/D, B/D or C/D, the latter of which is the natural state of the soil if not artificially drained (e.g., through tiling or ditching) while the former is if the soil is drained. In order to average the different components it was necessary to convert these letters into numbers, which, considering that the A through D naming scheme corresponds to increasingly wetter drainage conditions was accomplished by changing the groups from A through D to 1 through 4. Once a number was obtained for the HSG, it was treated as any other soil property and then rounded to the nearest integer and converted in the same manner to a letter. For those components with dual HSGs it was necessary to determine if that area was drained or not. A preliminary analysis was conducted using the SWAT SSURGO database which was already aggregated up to the mapunit level by the developers of SWAT for use in that model. \st{This was used instead of the unaltered SSURGO data in order to quickly reach a decision as to how the dual HSGs should be treated.} The soil mapunit polygons with dual HSGs were overlaid on the WRB landuse dataset. If more than half of the area in the mapunit was agriculture then it was assumed that the land was drained and the first HSG taken, conversely the mapunit was designated `D' if it was not majority agriculture. No mapunit polygon with dual HSG was found to be majority agriculture and thus when aggregating the SSURGO data all components with dual HSGs were given a `D' designation.

\textit{SWAT requires each soil type to have only one HSG, A, B, C or D. In SSURGO, several soil types were mapped as having dual HSGs, A/D, B/D and C/D, because that soil is of one HSG when drained (the first letter) and another when not drained (the D). Those mapunits with dual HSGs were overlaid onto the WRB landcover dataset and the proportion of land in agriculture was found. Arbitrarily, it was decided that if more than 50\% of the land was in agriculture then it would be assumed that that soil was drained, or else it was considered to be undrained. None of the dual HSG mapunits had a proportion of land in agriculture greater than 50\% and so all were given the D HSG.} 

The method of \citet{beaudette_aqp_2013} was used to aggregate multiple components within a mapunit. This method combines soil profile data by splicing the profiles into many thin slices and combining these slices together by a user-defined functioned. This aggregation methodology was used by \citet{gatzke_soilaggregation_2011} to aggregate SSURGO data for a SWAT hydrology study in California. The algorithm is implemented as the \texttt{slab} function in the aqp package in R  and fully described in \citet{beaudette_aqp_2013}. We used this algorithm to apply a depth weighted average to each horizon, while also weighting the percent composition of each component. This achieved a robust average of the soil properties for each horizon, while also accounting for differing compositions of each component. The depth and number of horizons of the aggregated soil profile produced by this algorithm must be specified before processing. The depth was calculated by using the weighted mean of the depths of the components, with the weights equal to the percent composition of each component. As the number of horizons was not seen to matter as much as the maximum depth, an arbitrary number of five horizons was chosen for the aggregation algorithm. 

The aggregation algorithm was not used on every mapunit. It was not necessary to aggregate mapunits with only one component. Additionally, the algorithm requires information on the horizon depths and so could not be applied to those mapunits without this information. Mapunits without this information included water bodies, urban land, landfills, and other miscellaneous disturbed areas. 

Using the above aggregation method 48,585 individual soil components were aggregated to 1,603 mapunits. Because the hydrologic	response unit (HRU) used in SWAT	is derived using unique combinations of land use, slope and soil types, this number of soil mapunits is still far too many for efficient computation \textbf{[More justification necessary?]} and so the second step of the soils data configuration was necessary to further reduce the number of soil types.

Other researchers have aggregated soil types by their taxonomic class \citep{gatzke_soilaggregation_2011} but Soil Taxonomy classifies largely based on soil morphology and not necessarily on relevant properties. We decided that the most relevant soils information to SWAT is hydrology data, specifically the hydrologic soil group (HSG), which has a large impact on the curve number calculation. With this consideration, aggregation was based around (and so preserved) the HSG of the mapunit. Groups of the same HSG were divided into smaller groups, hereafter known as clusters, of homogeneous soil properties, using a clustering algorithm. The mapunits within each of these clusters were then averaged together to create an average profile for that homogeneous set of soils. These averages were then used as the soil types for the HRU definitions and the SWAT modeling.

Each mapunit was placed into one of four groups according to its hydrologic soil group, A, B, C or D. To subdivide these groups further, a clustering algorithm was used to objectively and robustly create clusters of mapunits with homogeneous soil properties. For this purpose used Gaussian mixture models were used to assign mapunits to clusters. The mixture model algorithm we used was the \texttt{Mclust} function in the mclust package \citep{mclust_2012} in R. A mixture model is a probabilistic model for representing the presence of subpopulations within an overall population. In our case, the overall population would be the group of mapunits of like hydrologic soil groups (say all mapunits with an HSG of A), while the (unknown) subpopulations are the clusters of mapunits with similar distributions of soil properties (such as a cluster of sandier soils, shallow soils or slow saturated conductivity). This algorithm gives us control over the number of clusters of which to classify the mapunits while maintaining strong within group similarity and between group differences. Three clusters were chosen for each HSG in order to limit the number of HRUs in the SWAT model, resulting in twelve soil types.

Each of the 1603 mapunits had data of regarding the soil property values at each horizon. In this format it was thought  depth or number of horizons would negatively affect the clustering algorithm, causing groups to be entirely governed by these differences. To remedy this issue, depth weighted averages of the horizons were taken to derive one value per soil property for each mapunit; essentially collapsing the soil profile down to one aggregate horizon. Profile depth was still considered in the clustering algorithm by keeping the profile depth as a property and so in this way it is represented but not inordinately so. 

Several of the soil property fields of the SSURGO dataset did not seem to be populated or commonly had no data values. Further, we believe that several of the soil properties used by SWAT were not as important as others. These variables were pH, coarse fragments, albedo, salinity, and calcium carbonate concentration; these properties were not used in the clustering procedure so as not to change the clustering on spurious zero values or on properties that were not relevant. 

Not every mapunit was included in the clustering procedure. Those mapunits that did not have hydrologic soil group were not included, nor were mapunits that did not have information on the soil properties of the horizons. These included the same miscellaneous mapunits as were excluded from the aggregation algorithm used in aggregating components to the mapunit level: pits, landfills, urban or made land, and rock outcrops. These miscellaneous mapunits were grouped by their non-soil/no-property status. Water polygons were also not included in the clustering analysis.

The same soil profile aggregation algorithm \citep{beaudette_aqp_2013} used to aggregate several components together in the first part of the configuration was used to combine the soil profiles of a cluster into one composite soil profile. In this implementation each mapunit was given equal weight in the aggregation algorithm. Those mapunits designated as miscellaneous were aggregated into one soil profile as the other clusters were, while the water mapunits were not aggregated. The miscellaneous  grouping was assigned a hydrologic soil group by converting the letter designation into an ordinal integer (that is A, B, C, D to 1, 2, 3, 4) and the average was taken, rounded to the nearest integer, and converted by to the appropriate hydrologic soil group designation, which happened to be B. 

Soil Phosphorus concentrations were obtained through the University of Wisconsin Soil Testing Laboratory \href{http://uwlab.soils.wisc.edu/}. Soil phosphorus concentrations are aggregated by county by the soil laboratory for each year from 1974 to the present. We chose the annual average soil concentration at the beginning of the model spin-up period, 1991, to establish prior concentrations. Subbasin-level soil phosphorus concentrations were estimated by calculating an area-weighted average of intersecting counties within a subbasin. Because the phosphorus samples analyzed at the soils laboratory are strongly bias toward samples on agricultural fields, only agricultural HRUs were given the subbasin average concentration, while non-agricultural HRUs within each subbasin were given the default concentration (5 mg P/kg) and assumed to equilibrate over the 6-year model spin-up period. Soluble phosphorus concentrations were estimated as half (correspondence with Peter Vadas, needs citation) of the reported phosphorus using the Bray-1 method measured with a spectrophotometer. Organic phosphorus concentrations were estimated by assuming that phosphorus constitues 0.85\% of organic material measured by loss of weight upon ignition (correspondence with Phillip Barak, needs citation).

