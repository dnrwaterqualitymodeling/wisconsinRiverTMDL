\subsection{Soil Phosphorus}\label{sec:soil_phosphorus}
Soil Phosphorus concentrations were obtained through the University of Wisconsin
Soil Testing Laboratory \footnote{\url{http://uwlab.soils.wisc.edu/}}. Soil
phosphorus concentrations were aggregated by county by the soil laboratory for
each year from 1974 to the present. We chose the annual average soil
concentration nearest the beginning of the model spin-up period, 1995, to
establish prior concentrations. Subbasin-level soil phosphorus concentrations
were estimated by calculating an area-weighted average of intersecting counties
within a subbasin.
The soil testing laboratory receives almost exclusively agricultural soils so to reflect this bias in the soil phosphorus data, only agricultural HRUs were given the subbasin average concentration, while the non-agricultural HRUs were given SWAT's default concentration (5 mg P/Kg). This default concentration is assumed to equilibrate over the 6-year model spin-up period. 
Soluble phosphorus
concentrations were estimated as half of the reported phosphorus using the
Bray-1 method measured with a spectrophotometer \citep{vadas_validating_2010}.
Organic phosphorus concentrations were estimated by assuming that phosphorus
constitutes 0.85\% of organic material measured by loss of weight upon ignition
\citep{havlin_soilfertility_2005}. 
SWAT allows soil phosphorus values to be set at every soil horizon, in our case we changed the 
soil phosphorus values only for the first horizon, the rest were left at the default values.
