
\subsection{Evapotranspiration Equation}\label{sec:et_methods}

The method selected to model potential evapotranspiration is used across all subbasins within the model.  The three methods to choose from include the Hargreaves, the Penman-Monteith, and the Priestley-Taylor equations.  We determined which method would work best by evaluating the percent bias and the Nash-Sutcliffe model efficiency coefficient  when comparing modeled water yield to observed water yield at 20 sites across the basin. Without calibrating the initial model, Penman-Monteith outperformed the other two methods in both Nash-Sutcliffe coefficient and percent bias (Table \ref{table:et_method}). See Figure \ref{fig:et_methods} for maps of the quality metrics for each of the different evapotranspiration equations. The Penman-Monteith equation is an energy balance and aerodynamic formula that computes water evaporation from vegetated surfaces. The equation estimates evapotranspiration rates based on solar radiation, temperature, wind speed, and relative humidity.
