\subsection{HRU Definition}\label{sec:hru_definition}

The hydrologic response units in SWAT are defined by unique combinations of land use, soils and slope class and are unique for every subbasin. If there were unique combinations of each of these, there would many tens of thousands of HRUs and the model would take an impractical amount of time to run while have much functional redundancies within it, as many of the HRUs are negligibly small in area. To reduce the number of HRUs, ArcSWAT allows for the removals of small HRUs; by setting a threshold, a user can tell ArcSWAT to ignore HRUs if they are below a certain threshold and divide the area equally amongst the other HRUs. The first threshold is of land use, the second is soil type within any given land use, and the third threshold is the slope class within any given land use/soil combination. We found that thresholds of 1\%, 25\% and 50\% resulted in 4,400 HRUs; a manageable number for performing calibrations and yet detailed enough where there is little data resolution lost.