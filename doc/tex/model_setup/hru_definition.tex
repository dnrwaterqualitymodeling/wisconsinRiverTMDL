\subsection{HRU Definition}\label{sec:hru_definition}

The hydrologic response units in SWAT are defined by unique combinations of land use, soils, and slope class and are unique for every subbasin. 
If every combination were honored, there would be many tens of thousands of HRUs and the model would take an impractical amount of time to run while having a number of functional redundancies within it. 
To reduce the number of HRUs, ArcSWAT allows for the removal of small HRUs; by setting a minimum threshold for each landcover, soils, and slope class, we reduced the number of overall HRUs. 
First, if a given landcover covered less than 1\% of a subbasin, we excluded it and proportionally reallocated the remaining landcovers so that they would add to 100\%. 
Second, within the remaining landcovers, if a given soil type covered less than 25\% of a landcover class, it was excluded and reallocated in the same way as landcover. 
Finally, within the remaining landcover/soil combinations, if a given slope class covered less than 50\% of a landcover/soil combination, it was excluded and reallocated. This means that only the dominant slope class is used for a given landcover/soil combination.
This iterative method of exclusion resulted in 4,400 HRUs; a manageable number for performing calibrations and yet detailed enough where there is little data resolution lost. %Table \ref{tab:hru_table} shows the number of HRUs in each subbasin, the average and standard deviation of the area of the HRUs in each subbasin.