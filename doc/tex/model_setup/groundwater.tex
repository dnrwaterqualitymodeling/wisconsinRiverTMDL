\subsubsection{Groundwater Inflow (Baseflow)}\label{sec:baseflow}
	In SWAT, the relative contribution of streamflow as baseflow is determined by the ALPHA\_BF parameter and can be adjusted for each subbasin. An effort was made to regionalize this variable to account for the wide variations in baseflow conditions across the WRB. A model was constructed relating baseflow to local watershed characteristics and then this model was used to create a local value of ALPHA\_BF based upon local conditions.\footnote{The code used to carry out these and the following tasks is available here: \url{https://github.com/dnrwaterqualitymodeling/wisconsinRiverTMDL/tree/master/groundWater/baseFlow}}
	
	In order to construct a model relating baseflow contribution to watershed characteristics it was necessary to obtain observed values of baseflow. Bflow, a baseflow separation program, \citep{arnold_automated_1995} was used to determine baseflow from daily streamflow data. The observed streamflow data were retrieved from the USGS National Water Information System (NWIS) \citep{usgs_nwis_2014} on 27 August, 2014. All monitoring stations in Wisconsin that met the requirements of the Bflow separation program were used, excluding sites with upstream watersheds less than 50 km$^2$ or greater than 1,000 km$^2$ \citep{arnold_regional_2000}.
	
	This algorithm requires continuous daily observations of streamflow for at least one year, from which it determines the baseflow contribution from the hydrograph.	After the observed data were downloaded they were processed to ensure that only contiguous periods of streamflow of at least one year were used in the routine. For this analysis gaps of up to nine days were allowed in the record and still considered contiguous. If a monitoring site had a gap, or gaps, of longer than nine days, it was split at the gaps into separate records and each part assessed as to whether it contained at least a year of data. Therefore, it was possible for a monitoring site to have several periods of contiguous streamflow records. If a record spanned less than one year of data it was not used in the analysis.
	
	Each record of streamflow was analyzed using the SWAT Bflow filter \citep{arnold_automated_1995}. This smoothing algorithm produces several estimates of ALPHA\_BF, one for each successive pass of a smoothing filter; for every record, the final pass (third and smoothest) was used. For sites with multiple records, the ALPHA\_BF values were averaged, weighting the values by the length of the record. The end result was an ALPHA\_BF value for each monitoring station that satisfied the requirements for the Bflow filtering algorithm.  

	To estimate ALPHA\_BF for all SWAT subbasins in the WRB, we fit a multiple linear regression model to predict ALPHA\_BF using upstream watershed characteristics. Data regarding the landscape characteristics of the watershed for each monitoring station were retrieved from the Wisconsin DNR's Wisconsin Hydrography Dataset Plus (WHDPlus) \citep{wdnr_whdplus_2013}. Additionally, the EPA's Ecoregion boundaries level III were also used as a categorical predictor. We tested a suite of geologic, soil, and topographic watershed characteristics that could potentially affect baseflow by calculating Pearson correlation coefficients and visually analyzing scatterplots. We made an effort to avoid overfitting and multicollinearity by excluding collinear parameters based on correlation coefficients. The final model was selected based on $R^2$. We used residual plots to examine evidence of model bias. The best model used average slope of watershed, average permeability, and the EPA ecoregion boundaries. The ecoregion boundaries were used as a factor on the watershed slope term. The ecoregion term with slope was meant to allow for the expression of the effect of slope on the baseflow contribution in different regions in the WRB, e.g., different slope terms for the Driftless Area and the Central Sands.
	
	[Table of model terms?]
	
	This model was used to predict the ALPHA\_BF for every small watershed in the WHDPlus dataset. An area weighted average of these small watersheds was taken for each SWAT subbasin to aggregate the ALPHA\_BF predictions. These values were used to update ALPHA\_BF in the groundwater files for each subbasin.