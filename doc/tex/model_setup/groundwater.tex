\subsubsection{Groundwater Inflow (Baseflow)}\label{sec:baseflow}
	In SWAT, the relative contribution of streamflow as baseflow is determined by the ALPHA\_BF parameter and can be adjusted for each subbasin. An effort was made to regionalize this variable to account for the wide variations in baseflow conditions across the WRB. A regression model was fitted that relates baseflow to upstream watershed characteristics. Then this model was used to predict ALPHA\_BF at ungaged sites in the WRB\footnote{The code used to carry out these and the following tasks is available here: \url{https://github.com/dnrwaterqualitymodeling/wisconsinRiverTMDL/tree/model_setup_public/groundWater/baseFlow}}.
	In order to construct a model relating baseflow contribution to watershed characteristics it was necessary to obtain observed values of baseflow. The Baseflow Program \citep{arnold_automated_1995} was used to estimate baseflow from daily streamflow data. All monitoring stations in Wisconsin \citepalias{usgs_nwis_2014} that met the requirements of the Baseflow Program were used, excluding sites with upstream watersheds less than 50 km$^2$ or greater than 1,000 km$^2$ \citep{arnold_regional_2000}.

	This algorithm requires continuous daily observations of streamflow for at least one year, from which it determines the baseflow contribution from the hydrograph.	After the observed data were downloaded, they were processed to ensure that only contiguous periods of streamflow of at least one year were used in the routine. For this analysis gaps of up to nine days were allowed in the record and still considered contiguous. If a monitoring site had a gap(s), of longer than nine days, it was split at the gaps into separate records and each part assessed as to whether it contained at least a year of data. Therefore, it was possible for a monitoring site to have several periods of contiguous streamflow records. If a record spanned less than one year of data it was not used in the analysis.
	
	The Baseflow Program was run at each USGS gage site and a Bflow measure (analogous to the ALPHA\_BF parameter in SWAT) was computed \citep{arnold_automated_1995}. 
	This smoothing algorithm produces several Bflows, one for each successive pass of a smoothing filter. 
	The final pass (third and smoothest) was used in the regression model.
	For sites with multiple records, the Bflow values were averaged, weighting the values by the length of the record.
	The end result was a Bflow value for each monitoring station that satisfied the requirements for the Baseflow Program filtering algorithm. 

	The resulting ALPHA\_BF estimates from the Baseflow Program were site-specific, and thus were only valid for upstream subbasins. To parameterize ALPHA\_BF for ungaged subbasins, we fit a multiple linear regression model to predict Bflow using upstream watershed characteristics. 
	Data regarding the landscape characteristics of the watershed for each monitoring station were retrieved from WHDPlus \citep{wdnr_whdplus_2013}.
	Additionally, the Environmental Protection Agency's (EPA) Ecoregion boundaries level III were used as a categorical predictor. 
	We tested a suite of geologic, soil, and topographic watershed characteristics that could potentially affect baseflow by calculating Pearson correlation coefficients and visually analyzing scatterplots.
	We made an effort to avoid overfitting and multicollinearity by excluding collinear predictors. The final model was selected based on $R^2$ (Figure \ref{fig:alpha_bf_scatterplot}). 
	We used residual plots to examine evidence of model bias. 
	The best model (eq. \ref{eq:bflow}) used average slope of watershed, average permeability, and the EPA ecoregion boundaries. The ecoregion boundaries were used as a factor on the watershed slope term. 
	The ecoregion term with slope was meant to allow for the expression of the effect of slope on the baseflow contribution in different regions in the WRB ()e.g., different slope terms for the Driftless Area and the Central Sands).

	\begin{equation}
	\bm{A} = \beta_0 + 
		\beta_1 \bm{E_1} \bm{S} +
		\beta_2 \bm{E_2} \bm{S} + 
		\beta_3 \bm{E_3} \bm{S} +
		\beta_4 \bm{E_4} \bm{S} +
		\beta_4 \bm{P}
		\label{eq:bflow}
	\end{equation}
	
	Where $\bm{A}$ represents the SWAT ALPHA\_BF parameter controlling baseflow, $\bm{P}$ is average surface permeability of the watershed, $\bm{S}$ is the average slope of the watershed, while $\bm{E_x}$ are dummy variables denoting one of four ecoregion within the WRB (i.e., the slope term of $\bm{S}$ varies by ecoregion).
	
	This model was used to predict the ALPHA\_BF for every small watershed in the WHDPlus dataset. An area weighted average of these small watersheds was taken for each SWAT subbasin to aggregate the ALPHA\_BF predictions. These values were used to update ALPHA\_BF in the groundwater files for each subbasin. The resulting distribution of the ALPHA\_BF parameter in the WRB can be found in Figure \ref{fig:alpha_bf}.
	