%\subsubsection{Model Data}
\section{Model Data Files}
\label{sec:model_data}
The Wisconsin River Basin (WRB) SWAT model distributed for review contains all the files necessary to run the model. The ArcSWAT program (version 2012.10.15) was used to set up the SWAT project and so the file structure was determined by ArcSWAT. The SWAT executable that came bundled with this version of ArcSWAT was SWAT2012 revision 627.

Much of the model configuration was done with the R statistical package (files with a \texttt{.r} extension) and the Python scripting language (files ending with \texttt{.py}). These scripts can be found at the WDNR water quality modeling team's page on the GitHub\footnote{\url{https://github.com/dnrwaterqualitymodeling/wisconsinRiverTMDL/tree/model_setup_public}} website. Below is a brief description of the contents of each code folder and how the scripts inside were used. Included are references and links to sections for further description of methods. The titles of each folder are listed here as they appear on the GitHub site.

\begin{itemize}
\item \texttt{climate}: contains scripts pertaining to data processing for air temperature, precipitation, relative humidity, solar radiation, and a nearest neighbor algorithm for filling in missing data (see Section \ref{sec:climate_data})

\item \texttt{DEM}: scripts for processing the WRB elevation data from the National Elevation Dataset (NED)

\item \texttt{LandManagement}: the script \texttt{correlateCdlAndDatcpDairy.R} is for validating the predictions of the generalized rotation algorithm\footnote{The generalized rotation algorithm used was the algorithm from within the EVAAL tool. The script can be found here \url{https://github.com/dnrwaterqualitymodeling/EVAAL/blob/master/__EVAAL__.pyt}} and \texttt{GeneralizeMergeLandManagementLandCover.R} is for creating a map of the land management types of the WRB (see Section \ref{sec:ag_land_mgt})

\item \texttt{et}: the scripts contained in the folder pertain to investigating the proper equation for modeling evapotranspiration in the WRB (see Section \ref{sec:et_methods})

\item \texttt{groundWater}: the script \texttt{baseflow\_phosphorus.R} contains the processing for estimating the amount of phosphorus from groundwater and the scripts in the \texttt{baseFlow} folder contain the files for carrying out the processing and modeling for estimating baseflow contribution to streams (see Section \ref{sec:gwp} and \ref{sec:baseflow})

\item \texttt{hydrology}: \texttt{calculateRunoffUsingBaseflowSeparation.R} is a script for developing  basin-wide water balance estimates and formatting the outflows from the basin reservoirs

\item \texttt{landCover}: the script \texttt{mergeWwiWithNass2011.py} was used to reclassify and rasterize the Wisconsin Wetlands Inventory (WWI) into two classes: woody wetlands and herbaceous wetlands. The script \texttt{mergeCdlWithWetlandsCrpCranberries.py} merged the 2011 Cropland Data Layer (CDL) with the rasterized wetlands, a layer of Conservation Reserve Program (CRP) attributed field boundaries within the 2007 Common Land Unit dataset, and a layer of cranberry bogs developed by the Wisconsin DNR (see Section \ref{sec:land_cover})

\item \texttt{ponds}: these scripts were used in the processing and modeling to derive pond parameters for SWAT (see Section \ref{sec:ponds})

\item \texttt{soils}: these scripts  determine which hydrologic soil group (HSG) to assign to dual HSG map units, aggregate SSURGO data, and process the soil phosphorus data  (see Section \ref{sec:soils})

\item \texttt{updateParameters}: the script in this folder is used to update the ArcSWAT database with specific data derived in many of the scripts in these folders; they include:
	\begin{itemize}
	\item Management operations
		\begin{itemize}
		\item Crop rotations
		\item Plant harvest schedule
		\item Fertilizer type/amount/schedule
		\item Tillage type and schedule
		\end{itemize}
	\item Groundwater
		\begin{itemize}
		\item Baseflow contributions
		\item Baseflow phosphorus concentrations 
		\end{itemize}
	\item Reservoirs, physical properties and daily outflow
	\item Internally draining areas (i.e., SWAT wetlands and Ponds)
	\item Soil phosphorus concentrations	
	\end{itemize} 

\item \texttt{watershedAggregation}: scripts for aggregating the Wisconsin Hydrography Dataset (WHDPlus) watersheds up to the SWAT subbasins (see Section \ref{sec:sub_delineation})

\item \texttt{wetlands}: script for calculating wetland parameters  (see Section \ref{sec:wetlands})

\item \texttt{doc}: contains the raw files for the development and compilation of this document

\end{itemize}
