\section{Introduction}
This document is intended to describe the configuration of the Soil and
Water Assessment Tool (SWAT) \citep{arnold_swat_1994} for the estimation of streamflow and sources
of sediment and phosphorus in the Wisconsin River basin and serves to further define the watershed response model 
outlined in the Wisconsin River TMDL scope of work \citepalias{wdnr_wrtmdl_2013}. It is a technical
document that presumes the reader has a high level of familiarity with SWAT\footnote{\url{http://swat.tamu.edu/}} and
the Wisconsin River Basin (WRB) Total Maximum Daily Load (TMDL)\footnote{\url{http://dnr.wi.gov/topic/TMDLs/WisconsinRiver/}}.
It is written specifically to water quality
modelers to invite a critical review and ensure the quality of the resulting
estimations. The document was written after the model was configured, but before
it has been calibrated. Therefore, the methods described here will likely change prior
to the release of the final model given new developments during the calibration
phase of the project. The model itself is available for download at the following URL: \\

\url{ftp://dnrftp01.wi.gov/geodata/wrb_swat/WRB.zip} (512 MB)\footnote{If the download does not automatically begin or if you are prompted for a password, copy and paste the link into your browser.}\\

Due to the large amount of data required to configure the Wisconsin River SWAT model, the
products and data described in this document were created using a series of data 
processing scripts
that can be executed using either the Python or R programming languages, depending upon the script. These scripts
are described in Section \ref{sec:model_data}. To the
extent possible, these scripts have been written in a way that makes
the data processing transparent and reproducible. However, because the model is
not complete in configuration or documentation, in some cases the scripts are not
annotated to assist in interpretation. Therefore, the reader needs to have a
strong understanding of computer programming, Python and R syntax, statistics,
and spatial data processing in order to completely understand these scripts. These scripts are publicly available on the
website GitHub owned by the username
\texttt{dnrwaterqualitymodeling}\footnote{\url{https://github.com/dnrwaterqualitymodeling}}. 
These scripts can be viewed and downloaded
across the history of their development. However, it must be noted that the
original datasets used in the processing are not available with the scripts due
to limitations in data storage and transfer.

The basic model configuration is described in Section \ref{sec:basic_model_config}. 
This section describes the methods used to set up the minimum data requirements for a SWAT project. 
The ArcSWAT program was used to set up the basic SWAT model. It facilitated many of the preliminary spatial data processing tasks that are otherwise difficult to configure manually.
These steps included the overlay of the land use, soils, and slope layers to create the hydrologic response units (HRUs), and the creation of the input HRU files needed to run SWAT. Section \ref{sec:basic_model_config} outlines the data processing steps prior to HRU definition including the following:

\begin{itemize}
\item climate data processing (Section \ref{sec:climate_data})
\item SWAT subbasin delineation (Section \ref{sec:sub_delineation})
\item HRU definition (Section \ref{sec:hru_definition}), and the components of the HRU definition:
	\begin{itemize}
		\item land cover data specification (Section \ref{sec:land_cover})
		\item soil data aggregation (Section \ref{sec:soils})
		\item slope classification (Section \ref{sec:slope})
	\end{itemize}
\end{itemize}

The Wisconsin Department of Natural Resources (WDNR) SWAT project team has developed and configured datasets additional to those necessary to create and run a SWAT model. These additional datasets were created for a number of different, often overlapping, reasons as listed below. 
\begin{itemize}
\item The WRB drains an area of 23,700 km$^2$ and has significant variation in soils, geologic history, vegetation, and land use. Therefore, additional datasets were created to describe regionally specific physical and chemical processes (Sections \ref{sec:ag_land_mgt}, \ref{sec:soil_phosphorus}, \ref{sec:gwp}, and \ref{sec:baseflow}).

\item The WRB TMDL is a large modeling effort that integrates multiple modeling platforms. We discuss the delineation of urban boundaries that will be modeled separately using the WinSLAMM model (Section \ref{sec:urban}).

\item We are incorporating observed data where available. Specifically, we incorporated daily flows monitored at reservoir outlets.
We will also describe how we have selected monitoring sites for streamflow calibration. (Sections \ref{sec:reservoirs} and \ref{sec:streamflow_calibration_sites}).
 
\item Some model parameters were set during the configuration phase of the model development; 
this was done to save time during the calibration phase and to open these decisions for external review. 
These parameters may be adjusted during the calibration process, but the estimates set during the configuration phase likely represent better starting points than the default values of the parameters (Sections \ref{sec:reservoirs}, \ref{sec:et_methods}, \ref{sec:baseflow}, and \ref{sec:gwp}).
\end{itemize}

This document represents a description of the configuration for the SWAT model being used in the Wisconsin River Basin TMDL. 
It must be noted that the SWAT modeling component of the WRB TMDL remains a work in progress and the methods and data described and distributed at this stage may change at a future date because of issues and concerns that may develop during the calibration phase.


% At a minimum, the basic SWAT configuration includes data for the following:

% \begin{enumerate}
	% \item Subbasin delineations
	% \item Land cover
	% \item Soil
	% \item Topographic slope
% \end{enumerate}

%



\pagebreak

%%%% old

%The above datasets are those that are described in this document. Additional
%datasets will be added before the final calibrated model, particularly those
%that represent drivers of phosphorus export. Phosphorus loads will be the final
%step in model calibration; because phosphorus loads are dependent on streamflow
%and sediment loads, phosphorus must be calibrated independently from them. In
%addition, because most of the water-quality impairments in the basin are related
%to phosphorus loads, it is important to reserve some datasets to inform the
%final calibration rather than presuming the \textit{a priori} definition fully
%explains mechanisms of phosphorus export. Two examples of phosphorus-related
%datasets that are not described in this document, but will be added prior to
%final calibration of phosphorus loads, are soil phosphorus concentration
%estimates and instream baseflow phosphorus concentrations.

%%These elements are used to define discrete model components. Combinations of
% these elements form discrete units in the model that SWAT defines as hydrologic
% response units, or HRUs. Each of these elements have been configured for the
% Wisconsin River SWAT model, and HRUs have been classified---both of these
% processes are described in this document. HRUs only provide a simple, physical
% representation of water-quality drivers, and therefore a number of additional
% datasets were compiled to provide supplemental information to aid the accurate estimatation
% and calibration of the model to streamflow, sediment, and phosphorus.

% The additional datasets beyond just those for classifying HRUs supply
% information that are considered \textit{a priori} to describe regionally
% specific physical and chemical processes that impact streamflow and water quality.
% First, we compiled daily weather data for a large number of climate
% stations across the basin---although there is not enough topography within the
% basin to significantly impact weather, the Wisconsin River flows across enough
% latitude that the climate in the northern part of the basin is significantly
% different than that in the southern part. Therefore, a dense network of weather
% observations was necessary to adequately represent climatic geography.

% Additionally, the Wisconsin River spans several regions with very different
% hydrologic properties, mainly related to the surficial geology of the landscape.
% Specifically, internally draining areas on formerly glaciated topography
% were characterized using a terrain-based methodology. Groundwater flow
% contribution related to either highly permeable soils or hydrologic springs were
% characterized by modeling site-specific baseflow contribution with respect to a
% suite of watershed characteristics. 

% Finally, the Wisconsin River is a highly
% managed system with many impoundments created for either water storage or
% generating electricity. Most reservoirs within the basin have monitoring
% stations located at their outfall---we used these data as well as reservoir
% geometric properties (e.g., surface areas and volumes) to more accurately model
% streamflow, sediment, and phosphorus.