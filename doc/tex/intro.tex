\section{Introduction}
This document is intended to describe the configuration of the Soil and
Watershed Assessment Model (SWAT) for the estimation of streamflow and sources
of sediment and phosphorus in the Wisconsin River basin. It is a technical
document that presumes the reader has a high level of familiarity with SWAT and
the Wisconsin River TMDL. In fact, it is written specifically to water quality
modelers to invite a critical review to ensure the quality of the resulting
estimations. The document was written after the model was configured, but before
it has been calibrated. Therefore, the methods described here may change prior
to the release of the final model given new findings during the calibration
phase of the project. The model itself is available for download at the following URL: \\

\url{ftp://dnrftp01.wi.gov/geodata/wrb_swat/} \\

At a minimum, the basic SWAT configuration includes data for the following:

\begin{enumerate}
	\item Subbasin delineations
	\item Land cover
	\item Soil
	\item Topographic slope
\end{enumerate}

These elements are used to define discrete model components. Combinations of
these elements form discrete units in the model that SWAT defines as hydrologic
response units, or HRUs. Each of these elements has been configured for the
Wisconsin River SWAT model, and HRUs have been classified---both of these
processes are described in this document. HRUs only provide a simple, physical
representation of water-quality drivers, and therefore a number of additional
datasets were compiled to provide supplemental information to aid in calibrating
the model to streamflow, sediment, and phosphorus.

The additional datasets beyond just those for classifying HRUs supply
information that are considered \textit{a priori} to describe regionally
specific physical and chemical process that impact streamflow and water quality.
First and foremost, we compiled daily weather data for a large number of climate
stations across the basin---although there is not enough topography within the
basin to significantly impact weather, the Wisconsin River flows across enough
latitude that the climate in the northern part of the basin is significantly
different than that in the southern part. Therefore, a dense network of weather
observations was necessary to adequately represent climatic geography.
Additionally, the Wisconsin River spans several regions with very different
hydrologic properties, mainly related to the storage capacity of the landscape.
Specifically, internally draining areas due to recently glaciated landscapes
were characterized using a terrain-based methodology, and groundwater flow
contribution related to either highly permeable soils or hydrologic springs were
characterized by modeling site-specific baseflow contribution with respect to a
suite of watershed characteristics. Finally, the Wisconsin River is a highly
managed system with many impoundments created for either water storage or
generating electricity. Most reservoirs within the basin have monitoring
stations located at their outfall---we used these data as well as reservoir
geometric properties (e.g., surface areas and volumes) to more accurately model
streamflow, sediment, and phosphorus.

%The above datasets are those that are described in this document. Additional
%datasets will be added before the final calibrated model, particularly those
%that represent drivers of phosphorus export. Phosphorus loads will be the final
%step in model calibration; because phosphorus loads are dependent on streamflow
%and sediment loads, phosphorus must be calibrated independently from them. In
%addition, because most of the water-quality impairments in the basin are related
%to phosphorus loads, it is important to reserve some datasets to inform the
%final calibration rather than presuming the \textit{a priori} definition fully
%explains mechanisms of phosphorus export. Two examples of phosphorus-related
%datasets that are not described in this document, but will be added prior to
%final calibration of phosphorus loads, are soil phosphorus concentration
%estimates and instream baseflow phosphorus concentrations.

Due to extent of data required to configure The Wisconsin River SWAT model, the
products described above were created using a series of data processing scripts
that can be executed using either the Python or R programming languages. To the
extent possible, these scripts have been written in a way that intends to make
the data processing transparent and reproducible. However, because the model is
not complete in configuration or documentation, in most cases they are not
annotated to assist in interpretation. Therefore, the reader needs to have a
strong understanding of computer programming, Python and R syntax, statistics,
and processing of spatial data. These scripts are publicly available on the
website GitHub owned by the username
\texttt{dnrwaterqualitymodeling}\footnote{\url{https://github.com/dnrwaterqualitymodeling}}. These scripts can be viewed and downloaded
across the history of their development. However, it must be noted that the
original datasets used in the processing are not available with the scripts due
to limitations in data storage and transfer.

\pagebreak